% Created 2022-06-25 Sat 11:44
% Intended LaTeX compiler: pdflatex
\documentclass[11pt]{article}
\usepackage[utf8]{inputenc}
\usepackage[T1]{fontenc}
\usepackage{graphicx}
\usepackage{longtable}
\usepackage{wrapfig}
\usepackage{rotating}
\usepackage[normalem]{ulem}
\usepackage{amsmath}
\usepackage{amssymb}
\usepackage{capt-of}
\usepackage{hyperref}
\date{\today}
\title{}
\hypersetup{
 pdfauthor={},
 pdftitle={},
 pdfkeywords={},
 pdfsubject={},
 pdfcreator={Emacs 28.1 (Org mode 9.5.2)}, 
 pdflang={English}}
\begin{document}

\tableofcontents

\section{Classificazinoe autovalori}
\label{sec:orgccbea91}

prendi il solito sistema
\[\begin{cases}
\dot{x}(t) &= Ax(t) + Bu(t) \\
y(t) &= Cx(t)
\end{cases}\]

in laplace, con

\[ X(s) = \mathcal{L}\{x(t)\},\ Y(s) = \mathcal{L}\{y(t)\},\ U(s) = \mathcal{L}\{u(t)\} \]

si ottiene

\[\begin{cases}
sX(s) - x(0) &= AX(s) + BU(s) \\
Y(s) &= CX(s) + DU(s)
\end{cases}\]

che diventa

\[\begin{cases}
X(s) &= (sI-A)^{-1}x(0) \text{ (libera) } + (sI-A)^{-1}BU(s) \text{ (forzata)} \\
Y(s) &= C(sI-A)^{-1}x(0) \text{ (libera) } + C(sI-A)^{-1}BU(S) \text{ (forzata)}
\end{cases}\]

gli autovalori (non)controllabili si classificano per come lavorano
con l'ingresso, in particolare per "posso usare l'uscita per scazzare
col valore effettivo di questi cosi?".
Le parti del sistema che lavorano con l'ingresso (\(U(s)\)) sono

\begin{align*}
(sI-A)^{-1}BU(s) ( &= X_f (s)) \\
C(sI-A)^{-1}BU(s) ( &= Y_f (s))
\end{align*}

quindi gli autovalori controllabili saranno quelli che non partono
quando faccio \((sI-A)^{-1}\), in quanto quando questi non partono posso
modificare \(X_f (s)\) vede questi autovalori.

gli autovalori osservabili sono quelli che hanno un'effetto visibile
sull'uscita, l'uscita è

\[ Y(s) = C(sI-A)^{-1}x(0) \text{ (libera) } + C(sI-A)^{-1}BU(S) \text{ (forzata)} \]

Qui quelli che non partono quando faccio \(C(sI-A)^{-1}\) sono quelli
che avranno un effetto visibile sull'uscita, quindi quelli osservabili

(la parte con \(C(sI-A)^{-1}B\) non può aggiungere autovalori quindi nel
complessivo quelli restano)

quindi:
\begin{itemize}
\item non parte con \((sI-A)^{-1}B \to\) permette all'ingresso di
modificare lo stato \(\to\) \textbf{autovalore controllabile}
\item non parte con \(C(sI-A)^{-1} \to\) ha un'effetto visibile
sull'uscita \(to\) \textbf{autovalore osservabile}
\end{itemize}

inoltre:
\begin{itemize}
\item autovalori controllabili = radici di \(\varphi _c(s)\),
\textbf{polinomio caratteristico di controllo} del sistema
\begin{itemize}
\item per polinomii singolo ingresso = minimo comune multiplo di
di tutti i denominatori di \((sI-A)^{-1}B\)
\end{itemize}
\item autovalori osservabili = radici di \(\varphi _o(s)\),
\textbf{polinomio caratteristico di osservazione} del sistema
\begin{itemize}
\item per sistemi a singola uscita = minimo comune multiplo di
tutti i denominatori di \(C(sI-A)^{-1}\)
\item altimenti bla bla bla determinanti sottomatrici sticazzate
\end{itemize}
\end{itemize}

\subsection{Poli del sistema}
\label{sec:org501d5ad}
quando un autovalore non parte ne' con \(C(sI-A)^{-1}\) che con
\((sI-A)^{-1}B\) allora non scomparirà neanche con \(C(sI-A)^{-1}B\), vale
a dire la nostra cara vecchia \emph{funzione di trasferimento}.

i \textbf{poli del sistema}, saranno quindi quegli autovalori che sono \emph{non
solum} osservabili \emph{sed etiam} controllabili

nei sistemi SISO saranno le radici di \(G(s)\)

gli autovalori nascosti, (da non confondere con quelli non
osservabili, per quanto il nome lo faccia credere) saranno tutti gli
autovalori che \textbf{o} non sono osservabili, \textbf{o} non sono controllabili

\section{Retroazione algebrica sull'uscita}
\label{sec:orge9a97d6}

È una cosa molto limitata
\end{document}