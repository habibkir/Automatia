% Created 2022-06-25 Sat 11:24
% Intended LaTeX compiler: pdflatex
\documentclass[11pt]{article}
\usepackage[utf8]{inputenc}
\usepackage[T1]{fontenc}
\usepackage{graphicx}
\usepackage{longtable}
\usepackage{wrapfig}
\usepackage{rotating}
\usepackage[normalem]{ulem}
\usepackage{amsmath}
\usepackage{amssymb}
\usepackage{capt-of}
\usepackage{hyperref}
\usepackage{amsfonts}
\date{\today}
\title{}
\hypersetup{
 pdfauthor={},
 pdftitle={},
 pdfkeywords={},
 pdfsubject={},
 pdfcreator={Emacs 28.1 (Org mode 9.5.2)}, 
 pdflang={English}}
\begin{document}

\tableofcontents


\section{LTI TD}
\label{sec:org651d966}

abbiamo un sistema tempo-discreto descritto dalle solite equazioni per
lo stato e uscita

\begin{align*}
&x(t+1) = f(t,x(t),u(t)) \\
&y(t) = h(t,x(t),u(t)) \\
\end{align*}

Dato il titolo di questa sottosezione sappiamo che il sistema TD è
LTI, lineare-tempo-invariante, quindi

\begin{align*}
&x(t+1) = f(x(t),u(t)) \\
&y(t) = h(x(t),u(t)) \\
\end{align*}

con \(f(t)\) e \(h(t)\) funzioni lineari di f e u, vale a dire

\begin{align*}
&x(t+1) = Ax(t) + Bu(t) \\
&y(t) = Cx(t) + Du(t) \\
\end{align*}

con \(A\), \(B\), \(C\), e \(D\) matrici di qualisiasi dimensione faccia
tornare il problema.

Lo scopo di questa sottosezione è: \\
\textit\{Abbiamo lo stato, diciamo che ho anche l'input, se come varia
lo stato con stato e input allora so come varia lo stato in generale,

se diciamo di avere anche lo stato iniziale allora ho tutti gli stati
se ho tutti gli stati e so come va l'uscita allora ho tutto,
\emph{ANALISI SISTEMI BOIII}

\subsection{Formule et al}
\label{sec:org83a3627}
Andando di

\begin{align*}
&x(t+1) = Ax(t) + Bu(t) \\
&y(t) = Cx(t) + Du(t) \\
\end{align*}

Per qualche passo ci si ritrova con
\[
\text{mi fa fatica scriverlo in latex, facciamo dopo, ok?}\footnote{non lo fece dopo}
\]
Visto che qui siamo \emph{lineari} questo può essere diviso in 2
parti, che si sommano tra di loro
\begin{itemize}
\item La parte a cui non interessa l'input
\item La parte a cui interessa l'input
\end{itemize}

Si può agire in modo simile anche per l'uscita, ottenendo una forma
complessiva del sistema nella seguente forma
\[
\text{mi fa fatica farlo ora}
\]

Queste si dicono rispettivamente \textbf{risposta libera} e \textbf{risposta
forzata} e avranno le forme \emph{non mi va di farlo ora in latex} e \emph{non
mi va di farlo ora in latex}

\subsection{Risposte libere e forzate}
\label{sec:org7e924cc}

Evoluzione libera detta anche \emph{a ingresso nullo}, visto che è quello
che verrebbe fuori se l'input fosse nullo, quindi senza un cazzo di
contributo\par

L' evoluzione forzata è detta anche \emph{nello stato zero},
visto che è quello che verrebbe fuori se lo stato iniziale (e tutti i
successivi, data la linearità) fosse nullo, e non desse quindi alcun
contributo (mi sono scordato i conguingivi in seconda media, pardon)

Per la risposta hai la risposta libera, o \emph{evoluzione libera
dell'uscita}, e la risposta forzata, o \emph{evoluzione forzata
dell'uscita}\par

Se vuoi sembrare figo puoi dire \emph{principio di sovrapposizione degli
effetti} invece di \emph{conseguenze grazialcazzo della
linearità}

\subsection{Conseguenze grazialcazzo della linearità, TD}
\label{sec:org6e3b679}

fai finta che abbia scritto qualcosa

\section{LTI TC}
\label{sec:org9131e08}

un sistema LTI TC si descrive con

\begin{equation*}
\begin{cases}
\dot{x}(t) &= Ax(t) + Bu(t) \\
y(t) &= Cx(t) + Du(t)
\end{cases}
\end{equation*}

abbiamo la condizione iniziale, abbiamo l'input, facciamo il solito.

Qui le cose si fanno un pochino complicate, per semplicità
consideriamo il caso autonomo con una sola dimensione\footnote{sferico, nel
vuoto, e senza effetti relativistici}
quindi, con \(x(t) \in \mathbb{R}\) e \(a \in \mathbb{R}\), si ottiene

\[\dot{x}(t) = ax(t) \]
per non far arrabbiare i matematci tra i telespettatori possimao
specificare una condizione iniziale
\[x(0) = x_0 \]
essendo un'equazione differenziale di analisi 1 la risposta è \(e\), in
particolare
\[x(t) = e^{at} x_0\]
Magnifico e tutto, ma come faccio se ho una matrice? In situazioni
come queste è necessario ricordarsi che \emph{la matematica è una materia
in cui scopri cose nuove inventandoti merdate a caso}

\subsection{Abusi di notazione}
\label{sec:org1713a94}
La scoperta del numero \(e\) è stata fatta abbastanza ad hoc per
problemi di derivate \& Co.
Vediti il video di 3b1b, non so che dirti

\subsection{Roba}
\label{sec:org79b9f77}
qui mette i modi naturali, gli introduce in modo umano, bla bla bla
diagonalizzazione e da il contesto per sì e no il 70\% dei calcoli
fatti in questo corso, rivediti sta roba BENE, ma BENE, ma tanto BENE.
\end{document}
