% Created 2022-06-25 Sat 11:19
% Intended LaTeX compiler: pdflatex
\documentclass[11pt]{article}
\usepackage[utf8]{inputenc}
\usepackage[T1]{fontenc}
\usepackage{graphicx}
\usepackage{longtable}
\usepackage{wrapfig}
\usepackage{rotating}
\usepackage[normalem]{ulem}
\usepackage{amsmath}
\usepackage{amssymb}
\usepackage{capt-of}
\usepackage{hyperref}
\date{\today}
\title{}
\hypersetup{
 pdfauthor={},
 pdftitle={},
 pdfkeywords={},
 pdfsubject={},
 pdfcreator={Emacs 28.1 (Org mode 9.5.2)}, 
 pdflang={English}}
\begin{document}

\tableofcontents

\section{Regimi transitorii e permanenti}
\label{sec:orge662209}
Definita la risposta forzata si divide l'uscita forzata in due parti
\[y_f (t) = \mathcal{L}^{-1} \{G(s) U(s)\} = y_f^G(t) + y_f^U(t)\]
dove \(y_f^G (t)\) viene dai fratti semplici coi poli presi da G e
\(y_f^G (t)\) viene dai fratti semplici coi poli presi da U.

Se il sistema è esternamente stabile allora tutti i modi di \(G(s)\)
sono convergenti (poli con \(Re < 0\)), per questo \(y_g^G (t)\) è detta
risposta \emph{transitoria} del sistema, quello che resta allora sarà la
parte con i modi che non sono tutti convergenti, vale a dire quella
che ha i modi naturali presi dai poli di \(U(s)\) e si dice per
l'appunto risposta \emph{permanente} del sistema, i due casi più importanti
sono:
\begin{itemize}
\item \(u(t)\) pari a \(u_0 1(t)\)
\item \(u(t)\) pari a \(U_0 sin(\omega _0)\)
\end{itemize}

\subsection{Ingresso costante / gradino}
\label{sec:org358aea2}

facciamo sta scomposizione
\[ Y(s) = G(s) U(s) = Y_f^U(s) + Y_f^G(s)
=\frac{K_0}{s} + \text{roba che tanto } \to 0\]
è importante notare che il secondo termine \(\to 0\) solo quando \(G(s)\)
non ha poli in 0 a rovinarci tutto\par

\(\frac{K_0}{s}\) è l'unico elemento di questa formula con
\(\mathcal{L}^{-1}\) che non tende a 0, quindi vediamo quanto fa, per il
teorema dei residui :

\begin{align*}
K_0 &= \lim_{s \to 0} sY(s) \\
&= \lim_{s \to 0} s G(s) U(s)\\
&= \lim_{s \to 0} s G(s) \frac{U_0}{s}
\end{align*}

Per ipotesi abbiamo detto che \(G(s)\) non ha poli in 0, quindi non
esplode niente, risulta allora

\begin{align*}
K_0 &= \lim_{s \to 0} G(s) U_0 \\
&= U_0 G(0)
\end{align*}

che ci porta a

\[ Y_f (s) = \frac{K_0}{s} + \text{ roba che} \to 0 =
\frac{G(0)U_0}{s} + \text{ roba che} \to 0\]
\end{document}