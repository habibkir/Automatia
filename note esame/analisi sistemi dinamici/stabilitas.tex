% Created 2022-06-25 Sat 11:34
% Intended LaTeX compiler: pdflatex
\documentclass[11pt]{article}
\usepackage[utf8]{inputenc}
\usepackage[T1]{fontenc}
\usepackage{graphicx}
\usepackage{longtable}
\usepackage{wrapfig}
\usepackage{rotating}
\usepackage[normalem]{ulem}
\usepackage{amsmath}
\usepackage{amssymb}
\usepackage{capt-of}
\usepackage{hyperref}
\usepackage{amsfonts}
\usepackage{mathtools}
\author{Stocazzo}
\date{\today}
\title{Stabilità}
\hypersetup{
 pdfauthor={Stocazzo},
 pdftitle={Stabilità},
 pdfkeywords={},
 pdfsubject={},
 pdfcreator={Emacs 28.1 (Org mode 9.5.2)}, 
 pdflang={English}}
\begin{document}

\maketitle
\tableofcontents


\section{Cazz'è ?}
\label{sec:org0136295}

Stabilità vuol dire robustezza di come va il sistema (traiettoria del
sistema) rispetto a modifiche/ perturbazioni/ non essere propio esatto
di:
\begin{itemize}
\item \textbf{Input}
\item \textbf{Stato iniziale}
\item \textbf{Il sistema stesso}
\end{itemize}

Vale a dire di tutte le cose che possono influire sull'evoluzione/
traiettoria/ chiamala come vuoi del sistema.

In modo un po' più esatto sti cosi si chiamano:
\begin{itemize}
\item \textbf{Stabilità interna} : rispetto a varaizioni o scazzi dello stato
iniziale del sistema
\item \textbf{Stabilità esterna} : rispetto a variazioni o scazzi dell'input
che arriva al sistema
\item \textbf{Stabilità strutturale} : rispetto a variazioni o scazzi del
sistema stesso, com'è fatto (quindi per questo corso
rispetto alle classiche \(A\), \(B\), \(C\), e \(D\))
\end{itemize}

\section{Stabilità interna}
\label{sec:orgd9a4210}
\subsection{Tipi di stabilità}
\label{sec:org03e41ad}
\begin{itemize}
\item \textbf{Stabilità asintotica} : Il contributo della perturbazione
sparisce, converge a 0
\item \textbf{Stabilità marginale} : Il contributo della perturbazione
non va a 0, ma non diverge neanche
\item \textbf{Instabilità} : Il contributo della perturbazione diverge
\end{itemize}

\subsection{Mappa di transizione globale}
\label{sec:orgd8c689f}
La mappa di transizione globale (\(\Phi\)) è una descrizione completa
del comportamento sistema ottenuta buttandoci dentro tutto lo stato e
gli input del sistema, vale a dire:

\[\text{(t, stato iniziale, input) }
\xRightarrow{\Phi} \text{ stato del sistema} \]

questa viene usata per rendere un pochino più fattibile la discussione che segue
visto che a dire costantemente \emph{quello che fa se il sistema inizia così},
\emph{quello che fa con la configurazione cosà}\ldots{} si impazzisce tutti

la mappa di transizione è una caratteristica propia del sistema, quindi se hai
un sistema particolare

\[\begin{cases}
\dot{x}(t) &= Ax + Bu \\
y(t) &= Cx + Du
\end{cases}\]

ti ritrovi, per quanto visto ora, con la mappa

\[ \Phi (t,x_0,u) = e^{At}x_0 +
\int_{0}^{t} e^{A(t-\tau)} Bu(\tau) d\tau \]

\subsection{Stabilità interna}
\label{sec:org778ad8d}

Facciamo quindi che si studia la perturbazione rispetto allo stato
iniziale, questa sarà pari a

\[ \text{come va il sistema con $t$, $x_0 + \Delta x_0$, e $u$ (deviata)} -
\text{come va il sistema con $t$, $x_0$, e $u$ (nominale)} \] 

questa si può rappresentare con le mappe di transizione globale come

\[ \Phi(t,x_0+ \Delta x_0 ,u) - \Phi(t,x_0 ,u) \]

espandendo ste definizioni otteniamo

\begin{align*}
&e^{At}(x_0 + \Delta x_0) + \int_{0}^{t} e^{A(t-\tau)} bu(\tau) d\tau
- (e^{At}x_0 + \int_{0}^{t} e^{A(t-\tau)} bu(\tau) d\tau) = \\
& e^{At} \Delta x_0
\end{align*}

la perturbazione qui dipende solo da \(A\) e da \(\Delta x_0\), visto che allora il
sistema si comporta sempre allo stesso modo per tutte le perturbazioni di stato
iniziale possiamo parlare di \textbf{stabilità interna del sistema}

Da tabellina avremo

\begin{itemize}
\item \textbf{Asintoticamente stabile} $\iff$ \(e^{At}x_0\) converge sempre, quindi
\[\lim_{t \to \infty} e^{At} \Delta x_0 = 0\ \ \forall \Delta x_0\]
\item \textbf{Marginalmente stabile} $\iff$ \(e^{At}x_0\) sempre limitato, quindi
\[ \forall \Delta x_0\ \exists M :\
	  \lvert \lvert\ e^{At}\ \rvert \rvert\ < M\ \forall\ t > 0 \]
\item \textbf{Internamente Instabile} altrimenti, quindi se \(\exists \Delta x_0\)
che me lo fa esplodere in qualche modo
\end{itemize}

\subsection{Con modi naturali}
\label{sec:org3ddfb51}

ricordandoci che gli elemententi di \(e^{At}\) sono combinazioni lineari dei \textbf{modi
naturali del sistema} si ottiene che la stabilità interna dipende dai modi naturali

Quindi
\begin{itemize}
\item \textbf{Stabilità asintotica} $\iff$ tutto converge $\iff$ tutti i modi convergenti
\item \textbf{Stabilità marginale} $\iff$ tutto limitatao $\iff$ tutti i modi limitati
\item \textbf{Instabilità interna} $\iff$ almeno un modo divergente
\end{itemize}

Qui gli autovalori del sistema sono gli autovalori della matrice, quindi	  
\begin{itemize}
\item \textbf{Stabilità asintotica} $\iff$ tutti gli autovalori di \(A\) parte reale < 0
\item \textbf{Stabilità marginale} $\iff$ tutto con parte reale \(\le\) 0 \textbf{E} tutti quello
con parte reale = 0 hanno molteplicità = 1 come radici del polinomio minimo
\item \textbf{Instabilità interna} $\iff$ tutti gli altri casi, quindi \(\exists\) con
parte reale > 0 \textbf{O} \(\exists\) con parte reale = 0 e molteplicità > 1
\end{itemize}

Per gli esercizi avremo
\begin{enumerate}
\item Trova \(\varphi (s)\) polinomio caratteristico \(= det(sI - A)\)
\item Tutte radici < 0 ? asintoticamente stabile : continua
\item \(\exists\) radice di \(\varphi (s)\) > 0 ? internamente instabile : continua
\item tutte radici con parte reale \(\le\) 0
tutte quelle con parte reale = 0 hanno molteplicità 1? marginalmente
stabile : continua
\item trova il polinomio minimo, qui quelle con parte reale = 0 hanno
molteplicità 1? marginale : instabile
\end{enumerate}

\section{Risposta forzata e funzione di trasferimento}
\label{sec:orgc58d54b}

Al momento non sono provvisto di una quantità sufficiente di sbatti
per portare a compimento la scrittura della seguente sezinoe

\section{Criterii algebrici per la stabilità}
\label{sec:org3546042}

Abbiamo qualche rapporto tra stabilità e segni delle radici, in particolare
abbiamo visto che:

\begin{itemize}
\item Stabilità asintotica $\iff$ tutte le radici di \(\varphi (s)\) con Re < 0
\item Stabilità esterna $\iff$ tutte \(a(s)\) con Re < 0
\end{itemize}

Capire che radici ha un polinomio può non essere facilissimo
capire se sono tutte minori di 0 di solito è meno complicato

\subsection{Condizione necessaria e cartesio}
\label{sec:org49dcdea}

le radici hanno Re < 0 \(\to\) tutti i coefficienti sono non nulli e dello stesso
segno

per i polinomii di gradi 2 questa condizione è necessaria \textbf{e sufficiente},
questa $\iff$ aggine si chiama \textbf{Regola di Cartesio}

\subsubsection{Come la uso}
\label{sec:org509b6f0}

per \(n \leq 2\) ci butti quella e hai già fatto l'esercizio, almeno per quanto
riguarda il segno delle radici

per \(n > 2\) la possiamo usare come passo preliminare, se non passi quella non
passi e basta, quindi instabilità, per andare oltre si usa Routh Hurwitz

\subsection{Tabella di Routh}
\label{sec:org366ac0b}

fai una tabella

Criterio di Routh Hurwitz dice che

Tutte radici con Re < 0 $\iff$ tutti gli elementi della prima colonna della
tabella sono nonnulli con lo stesso segno

Generalizzazione di regola di Cartesio

Coso sopra, i due sopra, i due a destra destra

\section{Rappresentazione Ingresso Uscita (Battistelli.io)}
\label{sec:org58cfd75}

wee wee wa we weee
\end{document}
