% Created 2022-06-25 Sat 11:17
% Intended LaTeX compiler: pdflatex
\documentclass[11pt]{article}
\usepackage[utf8]{inputenc}
\usepackage[T1]{fontenc}
\usepackage{graphicx}
\usepackage{longtable}
\usepackage{wrapfig}
\usepackage{rotating}
\usepackage[normalem]{ulem}
\usepackage{amsmath}
\usepackage{amssymb}
\usepackage{capt-of}
\usepackage{hyperref}
\author{Tua madre}
\date{Giovedì scorso}
\title{Laplace}
\hypersetup{
 pdfauthor={Tua madre},
 pdftitle={Laplace},
 pdfkeywords={},
 pdfsubject={},
 pdfcreator={Emacs 28.1 (Org mode 9.5.2)}, 
 pdflang={English}}
\begin{document}

\maketitle
\tableofcontents


\section{Segnale causale}
\label{sec:org0e1f8fd}

Non esiste (= 0, per noi pezzenti) per \(t < 0\)

\section{Definizione}
\label{sec:org02019ea}

ecco la definizione

\subsection{Casi notevoli}
\label{sec:orgf5c6de1}

casi notevoli sono

\subsubsection{Il gradino}
\label{sec:orgeaf602e}

\subsection{Coso motliplicato per un esponenziale}
\label{sec:org391f8e4}

possiamo usare il prodotto per un esponenziale, con le formule di
eulero, per fare

\subsubsection{Seno}
\label{sec:org13c6999}

pari a

\subsubsection{Coseno}
\label{sec:org69a71b2}

pari a

\subsection{Derivata}
\label{sec:orgeeb3a89}

e con questa facciamo

\subsubsection{Rampe}
\label{sec:org1d299f0}

La rampa unitaria fa

La rampa quadratica fa

In generale, la rampa con sto grado fa
\end{document}