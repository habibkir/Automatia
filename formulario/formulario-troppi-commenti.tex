\documentclass{article}

\usepackage{amsmath}
\usepackage{amssymb}
\usepackage{amsfonts}

\begin{document}

\section{Sistemi di Controllo}

\subsection{Retroazione sullo Stato}

\begin{itemize}
\item Uscita controllore
  \[ u(t) = Hy^{0}(t) - Fx(t) \]
\item Sistema in ciclo chiuso
  \begin{align*}
    \mathcal{P}^{\ast} = &\begin{cases}
    \dot{x}(t) &= Ax(t) + Bu(t) \\
    y(t) &= Cx(t)
  \end{cases}
  \Rightarrow \begin{cases}
    \dot{x}(t) &= Ax(t) + B (Hy^{0}(t) - Fx(t)) \\
    y(t) &= Cx(t)
  \end{cases} \\
  \Rightarrow &\begin{cases}
    \dot{x}(t) &= (A-BF)x(t) + BHy^{0} \\
    y(t) &= Cx(t)
  \end{cases}
  = \begin{cases}
    \dot{x}(t) &= A^{\ast}x(t) + B^{\ast}y^{0}(t) \\
    y(t) &= Cx(t)
  \end{cases} \\
  \end{align*}
\item Polinomio caratteristico in ciclo chiuso
  \[ \varphi ^{\ast}(s) = \det(sI - A) = \det(sI - A + BF) \]%
\item Funzione di trasferimento in ciclo chiuso
  \[ G^{\ast}_{y^0 y} (s) = \frac{r(s)}{\varphi ^{\ast} (s)} \]
  con
  \[ r(s) = C Adj(sI-A)B \]
  nominatore della funzione di trasferimento di $\mathcal{P}$ normale
\end{itemize}

\subsection{Retroazione sul'Uscita}

\subsubsection{Retroazione Algebrica sull'Uscita}
è una retorazinoe algebrica sullo stato ma $F$ può solo essere
$K \times C$ con $K \in \mathbb{R}$
\begin{itemize}
\item Uscita controllore
  \begin{align*}
    u(t) &= -Ky(t) + H y^0 (t) \\
    &= -KCx(t) + Hy^0 (t)
  \end{align*}
\item Sistema in ciclo chiuso
  \[ \mathcal{P}^{\ast} = \begin{cases}
    \dot{x}(t) &= (A-BKC)x(t) + BHy^0 (t) \\
    y(t) &= Cx(t)
  \end{cases} \]
\item Funzione di Trasferimento in Ciclo Chiuso
  \[ G^{\ast} _{y^0 y} (s) = \frac{G(s)}{1+ KG(s)} \mathbf{H}\]
  Se fai
  \[ G(s) = \frac{b(s)}{a(s)} \]
  Ottieni
  \[ G^{\ast} _{y^0 y} (s)
  = \frac{G(s)}{1+KG(s)} \mathbf{H}
  = \frac{\frac{b(s)}{a(s)}}{\frac{a(s) + Kb(s)}{a(s)}} \mathbf{H}
  = \frac{b(s)}{a(s) + Kb(s)} \mathbf{H}
  = \frac{b(s)}{a^{\ast}(s)} \mathbf{H}\]
  vede solo l'uscita, non modifica gli autovalori nascosti
  $\varphi ^{\ast} (s) = \varphi_{normale} (s)$ \item Polinomio caratteristico in ciclo chiuso
  \[ \varphi ^{\ast} (s) = \varphi_{h} (s) a^{\ast} (s)
  = \varphi ^{\ast} (s) (a(s) + Kb(s)) \]
\end{itemize}

\subsubsection{Retroazione Dinamica sull'Uscita}
ora con $K$ e $H$ pari a $K(s)$ e $H(s)$
Scelta tipica $H(s) = H_f (s) K(s)$, spesso con
$H_f (s) = costante \in \mathbb{R}$, cose non segnate non cambiano
dall'algebrica

\begin{itemize}
\item Funzione di trasferimento in ciclo chiuso
  \begin{align*}
    G^{\ast} _{y^0 y}(s) &= \frac{G(s)}{1 + KG(s)} \mathbf{H(s)} \\
    &= \frac{G(s)K(s) \mathbf{H_f}}{1 + K(s)G(s)}
  \end{align*}
  Mettendo \textbf{non solum}
  $G(s) = \frac{b(s)}{a(s)}$ \textbf{sed etiam}
  $K(s) = \frac{q(s)}{p(s)}$ si ottiene
  \[ G(s) = \frac{b(s)q(s)}{a(s)b(s) + b(s)q(s)} \mathbf{H_f} =
  \frac{b^{\ast}(s)}{a^{\ast}(s)} \mathbf{H_f} \]
  anche questa retroazione, agendo sull'uscita, non tocca gli autovalori nascosti
  $\varphi ^{\ast} (s) = \varphi_{normale} (s)$
\item Polinomio caratteristico in ciclo chiuso
  \[ \varphi ^{\ast} (s) = \varphi_h (s) a^{\ast} (s)
  = \varphi _h (s) a(s)b(s) + b(s)q(s) \]
\end{itemize}

\subsubsection{Regolatore}

Se si riesce ad approssimare lo stato da fuori non dobbiamo preoccuparci dei
limiti della retroazione sull'uscita

\begin{itemize}
\item Osservatore di Luenberger
  \begin{align*}
    \mathcal{O} : \frac{d\hat{x}}{dt} &= A\hat{x} + B\hat{x} + L(y-C\hat{x}) \\
    &= A\hat{x} + B\hat{x} + L(C(x-\hat{x}))
  \end{align*}
\item Evoluzione dell'errore (errore si chiama $\epsilon$, $e$ sarebbe ambiguo)
  \begin{align*}
    \epsilon (t) &= (x(t) - \hat{x}(t)) \\
    \frac{d\epsilon (t)}{dt} &= \frac{d(x(t) - \hat{x}(t))}{dt}
    = \text{roba} = (A-LC)(\epsilon (t))
  \end{align*}
\end{itemize}
si considera l'errore $\epsilon (t)$ come parte dello stato del sistema con
l'osservatore in ciclo chiuso, quindi 
\begin{itemize}
\item Uscita della retroazione sullo stato approssimato
  \[ u(t) = -F\hat{x}(t) + Hy^0 (t) = -F(x(t) - \epsilon (t)) + Hy^0 (t)\]
\item Stato ed evoluzione completa del sistema in ciclo chiuso
  \[\begin{cases}
  \dot{x}(t) &= Ax(t) + Bu(t) \\
  y(t) &= Cx(t)
  \end{cases} \Rightarrow
  \begin{cases}
    \dot{x}(t) &= Ax(t) - BF(x(t) - \epsilon (t)) + BH y^0 (t) \\
    \dot{\epsilon}(t) &= (A-LC) \epsilon (t) \\
    y &= Cx
  \end{cases}\]
  per ``comodità'' si riscrive $\dot{x}(t)$ come:
  \[\dot{x}(t) = (A-BF)x(t) + BF(\epsilon (t)) + BH y^0 (t) =
  \begin{bmatrix} (A-BF) & BF \end{bmatrix}
  \begin{bmatrix} x(t) \\ \epsilon(t) \end{bmatrix}\]
  Prendendo ora un ``superstato'' fatto da
  $\begin{bmatrix} x(t) \\ \epsilon(t) \end{bmatrix}$
  si ottiene
  \[\begin{cases}
  \begin{bmatrix} \dot{x}(t) \\ \dot{\epsilon}(t) \end{bmatrix} &=
  \begin{bmatrix} (A-BF) & BF \\ 0 & (A-LC) \end{bmatrix}
  \begin{bmatrix} x(t) \\ \epsilon(t) \end{bmatrix} +
  \begin{bmatrix} BH \\ 0 \end{bmatrix} y^0\\
  y(t) &= \begin{bmatrix}C & 0\end{bmatrix}
    \begin{bmatrix} x(t) \\ \epsilon(t) \end{bmatrix}
  \end{cases}\]
  quell'abominio di matrice è come lo stato fa la derivata di se stesso, nota a
  noi ipse-dixitiani come la \textbf{matrice della dinamica} in ciclo chiuso,
  $\varphi (s)$ di un sistema di solito è il determinante della matrice della
  dinamica, quindi
\item Polinomio caratteristico in ciclo chiuso
  Quell'affare di matricione è \textbf{quadrato a blocchi}, quindi il
  determinante/ polinomio caratteristico sarà
  \[ \varphi ^{\ast} (s) = \det(sI - A + BF) \det(sI - A + LC) \]
\end{itemize}
\end{document}
