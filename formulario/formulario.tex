% Created 2022-06-22 Wed 11:30
% Intended LaTeX compiler: pdflatex
\documentclass[11pt]{article}
\usepackage[utf8]{inputenc}
\usepackage[T1]{fontenc}
\usepackage{graphicx}
\usepackage{longtable}
\usepackage{wrapfig}
\usepackage{rotating}
\usepackage[normalem]{ulem}
\usepackage{amsmath}
\usepackage{amssymb}
\usepackage{capt-of}
\usepackage{hyperref}
\usepackage{amsfonts}
\date{\today}
\title{}
\hypersetup{
 pdfauthor={},
 pdftitle={},
 pdfkeywords={},
 pdfsubject={},
 pdfcreator={Emacs 28.1 (Org mode 9.5.2)}, 
 pdflang={English}}
\begin{document}

\tableofcontents

Fonti usate: 

esercizii di riepilogo (non ha pubblicato un programma, che io sappia)
e/o slide e/o la cazzo di cane

\section{Trasformata Laplace}
\label{sec:org03f9a32}
\begin{itemize}
\item Definzione
\[\mathcal{L}\{f(t)\} = \int_{0}^{\infty} f(t) e^{-st} dt = F(s)\]
\end{itemize}

\subsection{Teoremi}
\label{sec:orgbac12db}
\begin{itemize}
\item Ritardo
\[\mathcal{L}\{e^{\lambda t}f(t)\} = F(s-\lambda)\]
\item Prodotto per \(t\) / derivata in frequenza
\[\mathcal{L}\{t f(t)\} = -\frac{d}{ds} F(s)\]
\end{itemize}

\subsection{Notevoli}
\label{sec:orgeb9d876}
\begin{itemize}
\item Gradino (ricavi con definizione)
\[ \mathcal{L}\{1(t)\} = \frac{1}{s} \]
\item Esponenziale causale (ricavi col teorema del ritardo)
\[ \mathcal{L}\{e^{\lambda t} 1(t)\} = \frac{1}{s- \lambda} \]
\item Seno causale (ricavi con l'esponenziale)
\[ \mathcal{L}\{\sin(\omega _0 t) 1(t)\} = \frac{\omega _0}{s^2 + \omega _0 ^2} \]
\item Coseno causale (ricavi con l'esponenziale)
\[ \mathcal{L}\{\cos(\omega _0 t) 1(t)\} = \frac{s}{s^2 + \omega _0 ^2} \]
\item Impulso di dirac (ricavi con definizione) (usata per funzioni semplicemente proprie)
\[ \mathcal{L}\{\delta (t)\} = 1 \]
\item Rampa unitaria (ricavi prodotto per t) (usata per poli nulli a molteplicità 2)
\[\mathcal{L}\{t 1(t)\} = -\frac{d}{ds} \mathcal{L}\{1(t)\}
	  = -\frac{d}{ds}(\frac{1}{s}) = \frac{1}{s^2}\]
\item Rampa parabolica (ricavi prodotto per t) (usata per poli nulli a molteplicità 3)
\[\mathcal{L}\{t^2 1(t)\} = \mathcal{L}\{t \times t 1(t)\} =
	  -\frac{d}{ds} (\frac{1}{s^2}) = \frac{2}{s^3}\]
\item Esponenziale \texttimes{} monomoio
(ricavi ripetendo derivate) (usata per poli non nulli a molteplicità multipla)
\begin{align*}
\text{forma generica } &\Rightarrow \frac{t^l}{l!} e^{at} 1(t) \\%
%
l=1 &\Rightarrow  \mathcal{L}\{t e^{at} 1(t)\} \\
& = -\frac{d}{ds} \mathcal{L}\{e^{at} 1(t)\} \\
& = -\frac{d}{ds} ( \frac{1}{s-a} ) \\
& = \frac{1}{(s-a)^2} \\ \\
%
l=2 &\Rightarrow \mathcal{L}\{\frac{t^2}{2} e^{at} 1(t)\} \\
& = \frac{1}{2} \mathcal{L}\{t \times t e^{at} 1(t)\} \\
& = -\frac{1}{2} \frac{d}{ds} \mathcal{L}\{t e^{at}1(t)\} \\
& = -\frac{1}{2} \frac{d}{ds} (\frac{1}{(s-a)^2}) \\
& = -\frac{1}{2} \frac{-2}{(s-a)^3} \\
& = \frac{1}{(s-a)^3}
\end{align*}
Andando per induzione ti ritrovi
\[ \mathcal{L}\{\frac{t^l}{l!} e^{at} 1(t)\} = \frac{1}{(s-a)^{l+1}}\]

(i fattoriali li mette perchè a ogni passo hai un'altra potenza, a ogni potenza devi
rifare la derivata che ti moltiplicare per l'intero dopo, hai tutti sti di interi e come
li togli? dividi per il fattoriale(!))
\end{itemize}

\section{Antitrasformata laplace / Analisi modale}
\label{sec:orge8458b1}
\subsection{Residui}
\label{sec:orgc9bb973}

\subsection{Risposta per poli\ldots{}}
\label{sec:orga21070b}
\begin{itemize}
\item Reali:
\begin{itemize}
\item Modo naturale \(e^{polo\ t}\), con \(t \geq 0\)
\end{itemize}
\item Complessi coniugati \(\sigma \pm j\omega\) (I complessi sono per forza coniugati qui, in
quanto radici di un polinomio con coefficienti reali)
 Detti:
\begin{itemize}
\item \(K\) e \(\overline{K}\) i residui corrispondenti ai poli, che saranno coniugati complessi
\item \(\alpha\) e \(\beta\) le parti reali e immaginarie di \(K\)
\end{itemize}
\[ e^{\sigma t} (2 \alpha \cos(\omega t) - 2 \beta \sin(\omega t)) 1(t) \]
\end{itemize}

\subsection{Poli a molteplicità multipla}
\label{sec:org845022e}
\begin{itemize}
\item Pari a 0 molteplicità \(l\)
\begin{itemize}
\item Laplace : fratti semplici \(\frac{K_{boh,1}}{s}\), \(\frac{K_{boh,2}}{s^2}\), \ldots{}, \(\frac{K_{boh,s}}{s^l}\)
\item Tempo : modi naturali \(1(t)\), \(t 1(t)\), \ldots{}, \(t^{l-1} 1(t)\)
\end{itemize}
\item Reali con molteplicità pari a \(l\)
\begin{itemize}
\item Laplace : fratti semplici \(\frac{K_{boh,1}}{(s-a)}\), \(\frac{K_{boh,2}}{(s-a)^2}\),
\ldots{}, \(\frac{K_{boh,l}}{(s-a)^l}\)
\item Tempo : modi naturali \(e^{at}\), \(t e^{at}\), \(t^2 e{at}\), \ldots{}, \(t^{l-1} e^{at}\)
\end{itemize}
\end{itemize}


\section{Sistemi LTI TC}
\label{sec:org4e59118}
\begin{itemize}
\item Definizione:
\[\begin{cases}
	  \dot{x}(t) &= Ax(t) + Bu(t) \\
	  y(t) &= Cx(t) + Du(t)
	  \end{cases} \]
per i sistemi SISO (Single Input, Single Output) \(B\) è un vettore colonna,
\(C\) è un vettore riga, \(D \in \mathbb{R}\)
\end{itemize}

\subsection{Evoluzione stato e uscita}
\label{sec:orgba817b5}
\subsubsection{Evoluzioni nel tempo}
\label{sec:org46addc2}
\begin{itemize}
\item Evoluzione stato nel tempo
\begin{itemize}
\item libera
\[x_l (t) = e^{At} x_0\]
\item forzata
\[x_f (t) = \int_{0}^{t} e^{A(t-\tau)} Bu(\tau) d \tau\]
\item complessiva
\[ x(t) = x_l (t) + x_l (t) =
	    e^{At} x_0 + \int_{0}^{t} e^{A(t-\tau)} Bu(\tau) d \tau \]
\end{itemize}

\item Evoluzione uscita nel tempo
\begin{itemize}
\item libera
\[y_l (t) = C x_l (t) = Ce^{At} x_0\]
\item forzata
\[y_l (t) = C x_f (t) + D u(t) =
	    \int_{0}^{t} C e^{A(t-\tau)} Bu(\tau) d \tau + D u(t) \]
\item complessiva
\[y(t) = y_l (t) + y_f (t) = C e^{At} x_0 +
	    \int_{0}^{t} C e^{A(t-\tau)} Bu(\tau) d \tau + D u(t) \]
\end{itemize}
\end{itemize}
\subsubsection{Evoluzioni in Laplace}
\label{sec:org170c11c}
\begin{itemize}
\item Evoluzione stato in Laplace
\begin{itemize}
\item libera
\[X_l (s) = \mathcal{L} \{e^{At} x_0\} = (sI - A)^{-1} x_0\]
\item forzata
\[X_f (s) = (sI - A)^{-1} BU(s)\]
\item complessiva
\[X(s) = X_l (s) + X_f (s) =
	    (sI - A)^{-1} x_0 + (sI - A)^{-1} BU(s) \]
\end{itemize}

\item Evoluzione uscita in Laplace
\begin{itemize}
\item libera
\[Y_l (s) = C X_l (s) = C(sI - A)^{-1} x_0\]
\item forzata
\[Y_f (s) = C X_f (s) + D U(s) = C(sI - A)^{-1} BU(s) + DU(s)\]
\item complessiva
\[Y(s) = Y_l (s) + Y_f (s) =
	    C(sI - A)^{-1} x_0 + C(sI - A)^{-1} BU(s) + DU(s) \]
\end{itemize}
\end{itemize}

\subsubsection{Funzione di trasferimento}
\label{sec:org5fb6766}
\[G(s) = \frac{Y_f(s)}{U(s)} = C(sI-A)^{-1}B + D\]
\end{document}