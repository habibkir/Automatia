% Created 2022-06-08 Wed 09:43
% Intended LaTeX compiler: pdflatex
\documentclass[11pt]{article}
\usepackage[utf8]{inputenc}
\usepackage[T1]{fontenc}
\usepackage{graphicx}
\usepackage{longtable}
\usepackage{wrapfig}
\usepackage{rotating}
\usepackage[normalem]{ulem}
\usepackage{amsmath}
\usepackage{amssymb}
\usepackage{capt-of}
\usepackage{hyperref}
\date{\today}
\title{}
\hypersetup{
 pdfauthor={},
 pdftitle={},
 pdfkeywords={},
 pdfsubject={},
 pdfcreator={Emacs 28.1 (Org mode 9.5.2)}, 
 pdflang={English}}
\begin{document}

\tableofcontents

\section{Boh}
\label{sec:org48e543d}

In questa lezione vediamo gli esercizi relativi al controllo in retroazione sull'uscita

\subsection{Retroazione algebrica}
\label{sec:org65cf6db}
vediamo prima il caso di una retroazine algebrica sull'uscita, quindi dovemo fare il
guadagno in feedback e il guadagno in feed forward

prima troavre polinomio caratteristico e funzione di trasferimento

visto che è un sistema di ordine 3 da già l'aggiogata

\subsubsection{Polinomio caratteristico e funzione di trasferimento}
\label{sec:orgecec2ea}

I calcoli sono gli stessi ma ho un sistema di ordine maggiore, evviva, sticazzi

trovo il polinomio caratteristico, questo è \(s^3 -1\), le radici di sto polinomio sono le
radici cubiche complesse dell'unità, vertici di un triangolo equilatero sul cerchio
unitario

con questo polinomio caratteristico e l'aggiogata e altra roba poi vado a trovarmi la
funzione di trasferimento del sistema

vediamo per quali valori di \(\alpha\) ci sono semplificazioni nella funzione di
trasferimento, la difficoltà è principalmente saper ragionare su polinomii parametrici e
vedere per quali valori ci sono semplificazioni in cosi tipo la funzione di trasferimento

per completare lo studio della funzione di trasferimento devo vedere anche [\ldots{}]

\subsubsection{Per quali \(\alpha\) il problema è ben posto}
\label{sec:orgebb3d5e}

Il problema di controllo è ben posto se e solamente se il polinomio \(\phi\) \_h(s), che si
calcola come rapporto tra polinomio caratteristico e denominatore di G(s), è un polinomio
asintoticamente stabile

abbiamo visto nella parte prima che per \(\alpha\) \(\neq\) -1 allora abbiamo G(s) messo bene,
altrimenti ho un autovalore in 1 nascosto, se ho un autovalore instabile nascosto allora
ho un problema di controllo mal posto, perchè non posso stabilizzare il sistema

per \(\alpha\) \(\neq\) -1 abbiamo \(\phi\) \_h(s) = 1, senza autovalori instabili, quindi il problema
di controllo è ben posto

le prime due domande dell'esercizio sono domande di studio preliminari

\subsection{Lascia stare sti parametri}
\label{sec:orga35002c}

mettiamo \(\alpha\) = 1

\subsubsection{vediamo per quali valori di K abbiamo stabilitas et limitata outputitas}
\label{sec:org00de219}

una cosa che conviene mettere in un formulario è come sono messi i polinomii
caratteristici e varie funzioni a seconda della legge di controllo

per fare questa cosa vediamo il polimiomio caratteristico in ciclo chiuso
trovata questa abbiamo \(a(s)\) e \(b(s)\), troviamo \(\phi\) *(s), bello bello
al variare di K vediamo la stabilità di sto polinomio

\(\phi\) *(s) è un polinomio di 3\textsuperscript{\circle} grado, quindi per studiare la stabilità facciamo
routh-hourbiz (o come cazzo si scrive)

visto che il guadagno in feedback è un parametro la difficoltà qui resta localizzata al
fatto di dover lavorare con polinomii parametrici

facciamo routh e applichiamo la condizione necessaria e sufficiente

stabile \iff tutti gli elementi della prima colonna hanno lo stesso segno
quindi \iff K > 0 \and 4K-1 > 0 \and \frac{4K^2 - 4K + 1}{K} > 0

quidni questi,

suppongo di avere K > 0, il rapporto potrebbe essere positivo se entrambi num e den sono >
0 o sono entrambi < 0, ma qui sappiamo che il denominatore (K) deve essere positivo,
quindi bla bla bla quindi tolgo il denominatore perchè tanto\ldots{}, quindi bla bla bla blero

risolviamo il sistema, e amen.

quindi va fatto uno studio parametrico sulla base del guadagno in feedback K.

\subsubsection{Progettare, se possiblite K e H per avere inseguimento perfetto et al}
\label{sec:orgf97c38f}

perndiamo la retroazione sull'uscita,
\[ u = -Ky ^ Hy^{\circle} \]
vogliamo K e H in modo da avere inseguimento perfetto e stabilità asintotica

vogliamo stabilità asintotica, visto prima, credo

e vogliamo \(G* _{y^{\circle}y} (0)\) = 1, quindi troviamo \(G* _{y^{\circle}y} (0)\) in
funzione di H e troviamo H giusto

\[ G* _{y^{\circle}y} (s) = \frac{b(s)}{a(s) + Kb(s)} H \]

abbiamo tutti gli ingredienti apparte H (K sarà un parametro, sticazzi), mettiamo che
\(Puttanaio \rvert _{k=0} = 1\), vediamo l'H che lo soddisfa, e fatto, avremo una H in
funzione di K che mi da inseguimento perfetto e belli belli.

\subsubsection{Oh shit}
\label{sec:org0d9fca2}

Adesso abbiamo un riferimento non più costante ma sinusoidale, bisogna applicare il
\emph{teorema della risposta in frequenza}

applichiamo questo \emph{teorema della risposta in frequenza}

$\backslash$[ y\textsubscript{f}\textsuperscript{Y}\textsubscript{0}(t) = 5(Re $\backslash${G* \_\{y\textsuperscript{\circle}y\} (j\(\omega\) \_0)$\backslash$} \(\sin\)(\(\omega\) \_0(t))
\begin{itemize}
\item Im$\backslash${G* \_\{y\textsuperscript{\circle}y\} (j\(\omega\) \_0)$\backslash$} \(\cos\)(\(\omega\) \_0(t))) 1(t) $\backslash$]
\end{itemize}

(o qualcosa del genere)

la frequenza del riferimento costante è unitaria, quindi avremo la funzione di
trasferimento per j\(\omega\) \_0 = j\(\omega\), la trovi e tanti saluti

(l'esame lo passate anche se il conto coi numeri complessi non finisce bene, magari non
prenderete 30 ma l'importante è impostarlo)

quindi prendiamo la parte reale e la parte immaginaria della risposta in frequenza
(\(G*_{y^{\circle}y}\)) e ci troviamo il regime permanente del sistema

applico il teorema della risposta in frequenza e ci trovo\ldots{}
a meno di errori di calcolo, ma il procedimento è questo
\end{document}
